\section{Methodology}

% General methodology
Our general methodology is heavily based on the approach presented by Zhuang et al. [REFERENCE TO IMPROVING MCTS]. Although the additional requirement of variable board sizes was added (and this does technically cause some problems since it implies a variable feature vector length), we doubt this would be a significant issue to deal with. Unfortunately, their paper hardly covers any technical details and their source code is very extensive and undocumented, so we decided that it would be infeasible to adapt/translate this code to our own use within a reasonable timeframe for this course. Instead, we designed a simpler board representation which is able to keep track of all chains but doesn't support the precise categorization from the paper. The rest of the strategy is largely unchanged: we generate partially completed games, let an exact solver label them based on their best outcome, train an artificial neural network using a feature vector based on chains and then use this network as heuristic at certain points during our Monte Carlo tree search. An important addition we made was to use an insight from the paper by Barker and Korf (REFERENCE TO SOLVING DOTS AND BOXES) to recognize provably optimal moves, which helped both our solver and MCTS.

\subsection{Board representation}
\subsection{Monte Carlo tree search (MCTS)}
	In order to create an intelligent agent for the game Dots and Boxes, we started by implementing a Monte Carlo tree search algorithm in Python. We based our implementation on the existing MCTS implementation of Dieter Buys\cite{DieterBuys:MCTS}, as this implementation is in Python, very clear to read and use and easy to adapt to our own needs.
	
	Later we switched to Java, because Python didn't perform as good as Java for the board representation with chains that we introduced in a later stage (see Section \ref{s:optimalMoves}). We translated the Python code we had at that moment to a Java implementation, and continued from there on with the Java code. Based in this code, we made some optimizations, which are described in the following sections.
	
	\subsubsection{Early termination of simulation}
	In the simulation step of the Monte Carlo tree search algorithm, random moves will be played until the end of a game is reached. This is done to find out at the end which player will win if these moves are played, and return this as a result of the simulation. However, it is sometimes possible to determine the winner of a game before the end of a game is reached. If one of the two players has captured more than half of the total boxes that can be captured ($rows * columns$), then this player is guaranteed to win the game. This means that the simulation can be stopped whenever the winner is decided, resulting in shorter simulations.
	
	\subsubsection{Reuse of the search tree}
	Most implementations of Monte Carlo tree search don't retain the search tree after the tree is used to get the best move for a given state. For every search for a move, a new search tree is created and deleted after a move is returned.
	However, the search tree still holds relevant information after a move is returned. This information can be reused when we search for a new move. 
	
	Implementation for keeping the search tree is fairly simple. The node which corresponds with the move that has been played is either already in the search tree as a child of the current root node, or was not yet expanded and thus isn't in the search tree. In the case that the corresponding node is already in the tree, this node becomes the new root node and its subtree is retained. In the other case, a new node is created for the state that was reached by playing the move and this node is used as the new root node.
	
	We don't expect the increase in performance of using this optimization to be large, because the branching factor of a game of Dots and Boxes is very high. A lot of information that was valuable for the deciding the previous move, is no longer relevant thanks to this high branching factor. Only the part of the tree that was explored in the direction of this particular node is still relevant for a next search process.

	\subsubsection{Small optimizations}
	
	We also did some small optimizations at this point, like decreasing the number of new (non-primitive) object creations, which led to a decrease in average time spent per iteration from $7.4*10^{-6}$s to $5.9*10^{-6}$s.
	
	\subsubsection{Optimal moves\label{s:optimalMoves}}
	% TODO bespreking chains
	
	The use of optimal moves, which can be derived from the chains on the board, allows us to reduce the search space drastically. We no longer use totally random moves in the simulation of gameplay, but only use random optimal moves if these are present. Also in the expansion of a node in the search tree, we only consider optimal moves, which reduces the branching factor.
	An important assumption at this point is that an opponent will play as an intelligent agent, and will not just play random moves. 
	
	\subsubsection{Avoiding bad moves}
	
	In early game, MCTS still makes fairly random decisions and will sometimes decide to "open up" a chain (turn an open chain/loop into one or more half-open/closed chains), which simply seems to help the opponent. We added a heuristic which tries to avoid these moves at almost all costs.
	
	\subsubsection{Increased search time}
\subsection{Artificial neural network}
\subsection{Solver}

To generate the data set for the ANN, we need to exactly calculate the outcome of certain board situations, which we accomplished using alpha-beta search. Like in the paper by Barker and Korf \cite{Barker:2012:SD:2900728.2900788}, we used optimal moves to limit the search space. As an optimization, we also tried to apply the killer heuristic \cite{killerHeuristic} and compared its performance on 1000 games (generated using the method from section 2.3.1), which led to a decrease from 0.01s per game to 0.009 per game. We're not sure if this change is even statistically significant, however since we don't have any evidence to believe the heuristic hurts either, we kept it regardless.