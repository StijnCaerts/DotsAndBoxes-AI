\section{Methodology}

% General methodology
Our general methodology is heavily based on the approach presented by Zhuang et al. [REFERENCE TO IMPROVING MCTS]. Although the additional requirement of variable board sizes was added (and this does technically cause some problems since it implies a variable feature vector length), we doubt this would be a significant issue to deal with. Unfortunately, their paper hardly covers any technical details and their source code is very extensive and undocumented, so we decided that it would be infeasible to adapt/translate this code to our own use within a reasonable timeframe for this course. Instead, we designed a simpler board representation which is able to keep track of all chains but doesn't support the precise categorization from the paper. The rest of the strategy is largely unchanged: we generate partially completed games, let an exact solver label them based on their best outcome, train an artificial neural network using a feature vector based on chains and then use this network as heuristic at certain points during our Monte Carlo tree search. An important addition we made was to use an insight from the paper by Barker and Korf (REFERENCE TO SOLVING DOTS AND BOXES) to recognize provably optimal moves, which helped both our solver and MCTS.

\subsection{Board representation}

\subsubsection{Chains}

One of the most basic concepts in "Dots and boxes" strategy is the concept of chains, so we decided to create a board representation which explicitly keeps track of these. We define the valence of a box as the amount of lines (edges) around it. We say two boxes are connected if they are adjacent and have no edge in between. We define a chain as a series of connected boxes with valence two or three. All boxes with valence 2 or 3 are part of exactly one chain, all other boxes are part of no chains. Each chain also has a type:
\begin{itemize}
\item Closed: starts and ends in a box of valence three.
\item Half-open: starts with a box of valence three and ends with a box of valence two. \item Open: starts and ends with a box of valence two but doesn't connect back to itself.
\item Loop: starts and ends with a box of valence two but does connect back to itself.
\end{itemize}
This categorization is important, since closed and half-open chains both give rise to some optimal moves, while open chains and loops have the potential to become half-open and closed chains respectively.

\subsubsection{Data structures}

Every move, our board updates several data structures:

\begin{itemize}
\item edges: a 2D array tracking what edges have been filled in.
\item valence: a 2D array tracking the valence of each box.
\item chainAt: a 2D array tracking what chain (or no chain) every box is part of.
\item chains: keeps track of all the chains currently on the board. Each chain has a list of its member boxes, a size and a type. Although we sometimes iterate over this (like when calculating optimal moves), we opted to use a hash-set instead of a dynamic array since we also often have to add/remove elements.
\item state: tracks the board state. This is initialized to START, goes to MIDDLE the first time when boxes have been filled in and there are no optimal moves on the board at the moment and goes to END one move afterwards.
\item movesLeft: tracks the amount of moves left in total.
\item movesLeftPerColumn: a 1D array tracking the amount of moves left per column. This is used for fast move iteration/random move generation.
\item optimalMoves: a 1D array which stores 0, 1 or 2 optimal moves. Although we re-calculate this completely (based on the chain data structures) every move, this isn't a big issue since it only requires iterating over (in the worst case) all chains, which is a pretty small number compared to the board size in almost all scenarios.
\item undoStack: a dynamic array which (if the board is recording undos) stores transactions which allow us to reverse every move.
\end{itemize}

\subsubsection{Registering moves}

\subsubsection{Undo}

\subsubsection{Legal move generation and iteration}

\subsubsection{Optimal moves}
\subsection{Monte Carlo tree search (MCTS)}
	In order to create an intelligent agent for the game Dots and Boxes, we started by implementing a Monte Carlo tree search algorithm in Python. We based our implementation on the existing MCTS implementation of Dieter Buys\cite{DieterBuys:MCTS}, as this implementation is in Python, very clear to read and use and easy to adapt to our own needs.
	
	Later we switched to Java, because Python didn't perform as good as Java for the board representation with chains that we introduced in a later stage (see Section \ref{s:optimalMoves}). We translated the Python code we had at that moment to a Java implementation, and continued from there on with the Java code. Based in this code, we made some optimizations, which are described in the following sections.
	
	\subsubsection{Early termination of simulation}
	In the simulation step of the Monte Carlo tree search algorithm, random moves will be played until the end of a game is reached. This is done to find out at the end which player will win if these moves are played, and return this as a result of the simulation. However, it is sometimes possible to determine the winner of a game before the end of a game is reached. If one of the two players has captured more than half of the total boxes that can be captured ($rows * columns$), then this player is guaranteed to win the game. This means that the simulation can be stopped whenever the winner is decided, resulting in shorter simulations.
	
	\subsubsection{Reuse of the search tree}
	Most implementations of Monte Carlo tree search don't retain the search tree after the tree is used to get the best move for a given state. For every search for a move, a new search tree is created and deleted after a move is returned.
	However, the search tree still holds relevant information after a move is returned. This information can be reused when we search for a new move. 
	
	Implementation for keeping the search tree is fairly simple. The node which corresponds with the move that has been played is either already in the search tree as a child of the current root node, or was not yet expanded and thus isn't in the search tree. In the case that the corresponding node is already in the tree, this node becomes the new root node and its subtree is retained. In the other case, a new node is created for the state that was reached by playing the move and this node is used as the new root node.
	
	We don't expect the increase in performance of using this optimization to be large, because the branching factor of a game of Dots and Boxes is very high. A lot of information that was valuable for the deciding the previous move, is no longer relevant thanks to this high branching factor. Only the part of the tree that was explored in the direction of this particular node is still relevant for a next search process.

	\subsubsection{Small optimizations}
	
	We also did some small optimizations at this point, like decreasing the number of new (non-primitive) object creations, which led to a decrease in average time spent per iteration from $7.4*10^{-6}$s to $5.9*10^{-6}$s.
	
	\subsubsection{Optimal moves\label{s:optimalMoves}}
	% TODO bespreking chains
	
	The use of optimal moves, which can be derived from the chains on the board, allows us to reduce the search space drastically. We no longer use totally random moves in the simulation of gameplay, but only use random optimal moves if these are present. Also in the expansion of a node in the search tree, we only consider optimal moves, which reduces the branching factor.
	An important assumption at this point is that an opponent will play as an intelligent agent, and will not just play random moves. 
	
	\subsubsection{Increased search time}
\subsection{Artificial neural network}

We tried to use artificial neural networks as a quick and somewhat accurate prediction of the outcome of a game a while before it ends by first training it on a labelled set of partially-completed games. The label indicates whether or not the current player will win (1), tie (0) or lose (-1). We always split our data set into a training and validation set using a 4:1 ratio. We also always used one hidden layer with as many neurons as the mean of the amount of neurons in the input and output layer (just one). As the activation function, we used the hyperbolic tangent. On a final note: we are aware that the two graphs presented in this section have a lower validation error than training error. Although this is strange, as far as we know it doesn't necessarily indicate a problem with the machine learning model.

\subsubsection{First approach}

First, we generated the games by letting two agents play random legal moves against each other, unless there are optimal moves (which are then played first), until at least 70\% of the edges have been filled and there are no optimal moves left. The feature vector used to represent the board to the ANN consisted of the current player's score, the other player's score and the number of open chains and loops of different lengths (chains of a length above the maximum are simply capped to that maximum in the feature vector). We don't consider half-open chains and closed chains since their presence implies the presence of optimal moves, which can be used first in MCTS before relying on a heuristic. Here are the results from a data set of 5000 games:

\begin{center}
\includegraphics[scale=0.65]{images/ann_rmse+accuracy_(5000_games).png}
\end{center}

This ANN has an accuracy of 62.6\% on the validation set and 59.4\% on the training set, which isn't very high. In fact, after analyzing the weights it turns out the ANN is doing little more than just making a guess based on the difference in scores. This heuristic wasn't very satisfactory to us.

\subsubsection{More realistic boards and chain parity}

In our second approach, we actually let two semi-intelligent agents (namely, agent 4 from section 3.1) play against each other to create more realistic board scenarios. The simulation was halted at the point where the game reaches the state MIDDLE, as described in section 2.1.2. Our main idea is that during MCTS, any node in middle-game is evaluated using this heuristic, which provides a better estimate for the desirability of moves during early game. In end-game, the heuristic won't be used anymore but this is fine since MCTS with the aforementioned improvements should be able to play fairly well at this point (it's just end-game, after all).

	Secondly, we tried to incorporate a more advanced piece of "Dots and boxes" strategy into our agent, namely the "chain rule" [REFERENCE TO \url{http://web.archive.org/web/20070627082933fw_/http://cf.geocities.com/ilanpi/math.html#proof}]. This rule states that, in order to force the opponent to open up the first chain in end-game (which usually decides the game), the first player should aim to have the parity of the number of open chains of length larger than two equal the parity of dots on the board (it's the opposite for the second player). As such, we added a value to our feature vector which is 1 if the current chain parity is beneficial to the current player or -1 if it isn't. In games marked with the state MIDDLE, the exact number of chains isn't entirely decided yet, so the chain rules can't be simply applied to the current chain parity. Therefore, we tried to let our ANN learn a model which gives us non-trivial accuracy. Here are the results from a data set of 1000 games:
	
\begin{center}
\includegraphics[scale=0.65]{images/ann_rmse+accuracy_(MCTS2_1000_games).png}
\end{center}

This gives us an accuracy of 73.5\% on the validation set and 69.25\% on the training set, which is a lot higher than the accuracy of the chain parity value on its own (51.1\%), which is why we decided to put this ANN into our agent.
\subsection{Solver}

To generate the data set for the ANN, we need to exactly calculate the outcome of certain board situations, which we accomplished using alpha-beta search. Like in the paper by Barker and Korf \cite{Barker:2012:SD:2900728.2900788}, we used optimal moves to limit the search space. As an optimization, we also tried to apply the killer heuristic [REFERENCE TO \url{https://en.wikipedia.org/wiki/Killer_heuristic}] and compared its performance on 1000 games (generated using the method from section 2.3.1), which led to a decrease from 0.01s per game to 0.009 per game. We're not sure if this change is even statistically significant, however since we don't have any evidence to believe the heuristic hurts either, we kept it regardless.