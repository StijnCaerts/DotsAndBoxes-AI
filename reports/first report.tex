\documentclass[11pt,a4paper]{article}
\usepackage[utf8]{inputenc}
\usepackage[T1]{fontenc}
\usepackage{lmodern}
\usepackage{amsmath}
\usepackage{amsfonts}
\usepackage{amssymb}
\usepackage{graphicx}
\usepackage{fullpage}

\renewcommand{\familydefault}{\sfdefault}

\author{Wouter Baert \& Stijn Caerts}
\title{Machine Learning: Project \\ \small{First report}}

\begin{document}
	\maketitle
	\section{Literature}
	\section{Pipeline}
	Summary of strategy in paper. We aim to test other strategies as well, but this is the main pipeline we want to study first.
	State representation: use game specific knowledge of chains
	Learning: ANN
		input: board state, use minimax search to find score with an optimal strategy for both players
		output: "outputs values between -1 and 1 where 1 means that
		the current player will most certainly win and -1 means that
		the player will surely lose."
	Playing strategy: MCTS
		use of ANN in simulation step
	\section{Research questions}
	\subsection{Which strategy for game playing do you use?}
	Reading of research papers, implement different strategies and compare results
	\subsection{Which data representation do you use?}
	Board representation of chains, such that not the whole board should be scanned before each move
	\subsection{Which machine learning model(s) do you use to represent the game state?}
	same as previous question?
	\subsection{How will you evaluate your solution?}
	comparison of our implementation(s) with other implementations than our own
	
	\subsection{What is the computational and memory cost of preprocessing, learning and evaluating?}
	Analysis of experimental results for different problem sizes.
	
	\subsection{How do you force a decision within a given time limit?}
	Restrict number of simulations in MCTS
	
	\subsection{Can you represent and learn to recognize the concept of chains, a popular strategy in Dots-and-Boxes?}
	The pipeline we had in mind explicitly uses chains in the data representation. For black-box systems we would test the system by presenting it with board states where in order to win it needs to exploit the presence of chains.
	
	\subsection{How does your best game playing strategy compare to your other strategies (performance)?}
	Compare our different strategies by competing with each other.
	
	\subsection{What is the performance/time/space trade-off between your different strategies?}
	Evaluate results in function of the problem size.
	
\end{document}